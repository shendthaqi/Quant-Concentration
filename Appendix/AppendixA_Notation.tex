\section{Notation}

\begin{remark}[Notation]
    Banach spaces are usually denoted \( \mathcal{X} , \mathcal{Y}, \mathcal{Z}\), its norm is often denoted \( \Norm{\cdot}_{} \). \\
    Hilbert spaces are usually denoted \( \mathcal{H} , \mathcal{K}\), its scalar product is often denoted \( \angled*{\cdot;\cdot} \). We use the convention, that the scalar product is linear in the first argument and anti-linear in the second argument. 
    Vectors (not operators) are usually denoted \( x,y,z \). Functionals in the dual \( \mathcal{X}^{*} \) are denoted \( f,g,h \) (if not in Hilbert space). 
    Scalars are often denoted \( \alpha,\beta,\gamma \). Constants that act as bounds are often denoted \( a,b,c \). 
    Positive numbers that can be interpreted as time or a variable that runs from 0 to \( \infty  \) are often denoted \( s,t \). Stopping times are denoted \( \tau, \sigma, \rho \)
    Operators on \( \mathcal{L}(\mathcal{X}) \) are usually denoted \( S,T \). Projections are denoted \( P,Q \), unitaries are denoted \( U,V \). 
    Borel-measurable spaces are often denoted \( (\mathcal{X},\mathfrak{B}) \). 
    Probability measure spaces are often denoted \( (\Omega , \mathfrak{F}, \mathbb{P}) \) 
\end{remark}

