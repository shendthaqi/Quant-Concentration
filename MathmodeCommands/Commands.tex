% Here, we set the style for theorems, definitions, etc.
\newtheoremstyle{note}% ⟨name ⟩ {3pt}% ⟨Space above ⟩1 {3pt}%⟨Space below ⟩1
  {6pt}{6pt}
  {\upshape}% ⟨Body font ⟩
  {}% ⟨Indent amount ⟩2
  {\bfseries}% ⟨Theorem head font⟩
  {.}% ⟨Punctuation after theorem head ⟩
  {\newline}% ⟨Space after theorem head ⟩3
  {}% ⟨Theorem head spec (can be left empty, meaning ‘normal’)⟩


%Sets, events with conditional delimiter  
\DeclarePairedDelimiterX{\set}[1]{\{}{\}}{\setargs{#1}}
\NewDocumentCommand{\setargs}{>{\SplitArgument{1}{;}}m}
{\setargsaux#1}
\NewDocumentCommand{\setargsaux}{mm}
{\IfNoValueTF{#2}{#1} {#1\,\delimsize|\,\mathopen{}#2}}%{#1\:;\:#2}

\DeclarePairedDelimiterX{\rounded}[1]{(}{)}{\setargs{#1}}
\DeclarePairedDelimiterX{\squared}[1]{[}{]}{\setargs{#1}}
\DeclarePairedDelimiterX{\curled}[1]{\{}{\}}{\setargs{#1}}
\DeclarePairedDelimiterX{\angled}[1]{<}{>}{\setargs{#1}}


% Norm with three vertical lines
\newcommand{\Norm}[1]{{\left\vert\kern-0.25ex\left\vert\kern-0.25ex\left\vert #1 
    \right\vert\kern-0.25ex\right\vert\kern-0.25ex\right\vert}}



% symbol for "compactly contained in"
\newcommand{\ssubset}{\subset\joinrel\subset}

\setlist[description]{leftmargin=\parindent,labelindent=\parindent}



\theoremstyle{note}
\newtheorem{prop}{Proposition}[section]
\newtheorem{thm}[prop]{Theorem}
\newtheorem{lem}[prop]{Lemma}
\newtheorem{defn}{Definition}[section]
\newtheorem{corl}[prop]{Corollary}
\newtheorem{exm}{Example}
\newtheorem*{remark}{Remark}
\newtheorem{fact}[prop]{Fact}
\newtheorem*{nota}{Notation}
\newtheorem*{exer}{Exercise}






% arrows
\newcommand\myxrightarrow[2][]{\xrightarrow[{\raisebox{1.25ex-\heightof{$\scriptstyle#1$}}[0pt]{$\scriptstyle#1$}}]{#2}}

\newcommand\myxrightharpoonup[2][]{\xrightharpoonup[{\raisebox{1.25ex-\heightof{$\scriptstyle#1$}}[0pt]{$\scriptstyle#1$}}]{#2}}

% integral with letter superimposed

\DeclareMathOperator{\sint}{\ensurestackMath{\stackinset{c}{0pt}{c}{0pt}{\mathrm s}{\displaystyle\int}}}
\newlength\correct
\settowidth{\correct}{\ensuremath{\displaystyle\int}}

\DeclareMathOperator{\wint}{\ensurestackMath{\stackinset{c}{0pt}{c}{0pt}{\mathrm w}{\displaystyle\int}}}

