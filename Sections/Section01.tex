\newpage

\section{Operators on a Hilbert Space}
 I need to introduce theory for compact operators and specifically for Schatten-p operators. 
Let \( \mathcal{H} \) be a separable Hilbert space. Therefore, we can assume that there is a sequence \( (e_n)_n \) that forms an orthonormal basis. We define projections associated with this sequenc. For this purpose let \( N\in \mathbb{N} \), we define \( P_N \in \mathcal{L}(\mathcal{H}) \) by 
\[ P_Nx = \sum_{n=1}^{N} \angled*{x;e_n}e_n. \]



\begin{thm}[]
    Let \( T\in \mathcal{L}(\mathcal{H})   \) be a compact operator.
    Then,
    \begin{enumerate}[1)]
      \item \( \lim_{N \to \infty} \norm{T-P_NT}_{}=0 \);
      \item \( \lim_{N \to \infty} \norm{T-TP_N}_{}=0 \);
      \item \( \lim_{N \to \infty} \norm{T-P_NTP_N}=0 \).
    \end{enumerate}
    
\end{thm}


\begin{thm}[Spectral theorem for compact operators]
    
\end{thm}


\begin{defn}[Schatten p-norm]
    For \( T \in \mathcal{L}(\mathcal{H}) \), where \( (\nu_n)_n \) is the decreasing sequence of singular values, we define 
    \[ \norm{T}_{p} := \left(\sum_{n=1}^{\infty} \nu_n^{p}\right)^{1/p} \]
    if the right-hand sinde is finite. The space of such operators is denoted as \( \mathcal{S}_p(\mathcal{H}) \), \( \norm{\cdot}_{p} \) is a norm on \( \mathcal{S}_p(\mathcal{H}) \).
\end{defn}


\begin{lem}[]
    Let \( A \in \mathcal{S}_{p}\left(\mathcal{H}\right) \). Then
    \[ \norm{A}_{p} \geq \Norm{A}_{}. \]
\end{lem}


\begin{thm}[]
    Let \( A \in \mathcal{S}_p(\mathcal{H}) \). Then
    \[ \lim_{N \to \infty} \norm{P_NT-T}_{1}=0 .\]
\end{thm}

\begin{proof}
    We have 
    \[ \norm{A-P_NAP_N}_{1} \leq \norm{A-P_NA}_{1} + \norm{P_NA-P_NAP_N}_{1} \leq \norm{A-P_NA}_{1}+ \norm{A-AP_N}_{1} .\]

    1. If \( A \) is a rank-1 operator, then \( Ax= \alpha \angled*{x;f}g \) for some normalized \( f,g \in \mathcal{H} \), \( \alpha \in \mathbb{C} \). Then, for any orthonormal basis \( (f_n)_n \),
    \begin{align*}
      \norm{A-AP_N}_{1} &= \operatorname{tr} \abs{ A(I-P_N)} \\
                        &= \sum_{n=1}^{\infty} \abs{\angled*{A(I-P_N)f_n;f_n}} \\
                        &= \sum_{n=1}^{\infty} \abs{\angled*{A\left(\sum_{m >N}^{} \angled*{f_n;e_m}e_m\right);f_n}} \\ 
                        &= \sum_{n=1}^{\infty} \abs{ \angled*{ \angled*{\sum_{m>N}^{} \angled*{f_n;e_m}e_m;f}g;f_n}} \\
                        &= \sum_{n=1}^{\infty} \abs{ \angled*{ \angled*{\sum_{m>N}^{} \angled*{e_n;e_m}e_m;f}g;e_n}} \\
                        &= \abs{ \angled*{ \underbrace{\angled*{\sum_{m >N}^{} e_m;f}}_{\to 0}g;e_m}} \to 0.
    \end{align*}  

   2. If \( A \) is finite-rank operator, then 
   \[ Ax= \sum_{m=1}^{M} \alpha_m \angled*{x; f_m} g_m \equiv \sum_{m=1}^{M} A_m x .\] 
   Then, by triangle inequality 
   \[ \norm{A-P_NA}_{1} \sum_{m=1}^{M} \norm{A_m-P_NA_m}_{1} \to 0 .\]
   3. If \( A \) is infinite-rank operator, then
   \[ Ax = \sum_{m=1}^{\infty} \alpha_m \angled*{x;f_m}g_m \equiv \sum_{m=1}^{\infty} A_mx. \]
   Then, 
   \[ \norm{A-P_NA}_{1} \leq \underbrace{ \norm{\sum_{m=1}^{M} \left(A_m-P_NA_m\right)}_{1}}_{I}+ \underbrace{\norm{\sum_{m=M+1}^{\infty} \left(A_m-P_NA_m\right)}_{1}}_{II} .\]
  Consider the second term: For large \( M \) regardless of \( N \)
  \[ II \leq 2 \norm{\sum_{m=M+1}^{\infty}A_mx}_{1} = 2 \sum_{m=M+1}^{\infty} \abs{\alpha_m} \leq \frac{1}{2} \epsilon \]
  Consider the first term: For \( N \) large and dependent on \( N \) as seen before
  \[ I \leq \frac{1}{2}\epsilon. \]
  We treat the term \( \norm{A-AP_N}_{1} \) similarly.
\end{proof}

\begin{corl}[]
    Let \( A \in \mathcal{S}_p(\mathcal{H}) \). Then,
    \[ \lim_{N \to \infty} \norm{A-P_NAP_N}_{p}=0. \]
\end{corl}

